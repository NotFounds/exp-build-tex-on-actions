\documentclass[12pt]{jsarticle}
\begin{document}

\title{タイトル}
\author{著者}
\maketitle

\section{節}
引用例\cite{Sample}

\subsection{箇条書き}
\begin{itemize}
  \item item
\end{itemize}

\subsection{箇条書き(連番)}
\begin{enumerate}
  \item item
\end{enumerate}

\subsection{表}
\ref{tb:table}

\begin{table}[htbp]
  \centering
  \caption{a table}
  \label{tb:table}
  \begin{tabular}{cc}
    \hline
    foo & bar \\
    \hline \hline
    0 & 1 \\
    2 & 3 \\
    \hline
  \end{tabular}
\end{table}

\subsection{数式}
\subsubsection{epsilon-delta definition of the limit.}
Let $f(x)$ be defined for all $x \neq a$ over an open interval containing $a$. Let $L$ be a real number. Then
$$\lim_{x \rightarrow a} f(x) = L$$
if, for every $\epsilon > 0$, then exists a $\delta > 0$ such that if $0 < |x - a| < \delta$, then $|f(x) - L| < \epsilon$

\subsubsection{イプシロンデルタによる関数の連続性の定義}
実数値関数$f: \mathbf{R} \rightarrow \mathbf{R}$があり,$a$を含む開区間$I$において
$$\lim_{x \rightarrow a} f(x) = f(a)$$
であるとは,
$$\forall \epsilon > 0, \forall a \in I, \exists \delta > 0 \mbox{ s.t. } |x - a| < \delta \Rightarrow |f(x) - f(a)| < \epsilon$$

\bibliography{./refs.biiteitem}
\bibliographystyle{junsrt}

\end{document}
